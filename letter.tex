% Using mnras_template.tex
%
% LaTeX template for creating an MNRAS paper
%
% v3.0 released 14 May 2015
% (version numbers match those of mnras.cls)
%
% Copyright (C) Royal Astronomical Society 2015
% Authors:
% Keith T. Smith (Royal Astronomical Society)

%%%%%%%%%%%%%%%%%%%%%%%%%%%%%%%%%%%%%%%%%%%%%%%%%%
\documentclass[fleqn,usenatbib]{mnras}
\usepackage[T1]{fontenc}

% Allow "Thomas van Noord" and "Simon de Laguarde" and alike to be sorted by "N" and "L" etc. in the bibliography.
% Write the name in the bibliography as "\VAN{Noord}{Van}{van} Noord, Thomas"
\DeclareRobustCommand{\VAN}[3]{#2}
\let\VANthebibliography\thebibliography
\def\thebibliography{\DeclareRobustCommand{\VAN}[3]{##3}\VANthebibliography}

%%%%% AUTHORS - PLACE YOUR OWN PACKAGES HERE %%%%%
\usepackage{graphicx}	% Including figure files
\usepackage{amsmath}	% Advanced maths commands
\usepackage{amssymb}	% Extra maths symbols

%%%%%%%%%%%%%%%%%%%%%%%%%%%%%%%%%%%%%%%%%%%%%%%%%%

%%%%% AUTHORS - PLACE YOUR OWN COMMANDS HERE %%%%%


\newcommand{\vdag}{(v)^\dagger}
\newcommand\latex{La\TeX}
\newcommand{\Msun}{\,{\rm M}$_{\odot}$\,}
\newcommand{\Msunh}{\,{\rm M}$_{\odot}$\,\,\ifmmode h^{-1}\else $h^{-1}$\fi}
\newcommand{\kms}{\,{\rm km s}\ifmmode ^{-1}\,\else $^{-1}$\,\fi}
\newcommand{\Mpch}{\,{\rm Mpc}\,\ifmmode h^{-1}\else $h^{-1}$\fi}
\newcommand{\kpch}{\,{\rm kpc}\,\ifmmode h^{-1}\else $h^{-1}$\fi}
\newcommand{\kpc}{\,{\rm kpc}\,}

%%%%%%%%%%%%%%%%%%%%%%%%%%%%%%%%%%%%%%%%%%%%%%%%%%

%%%%%%%%%%%%%%%%%%% TITLE PAGE %%%%%%%%%%%%%%%%%%%

\title[Cosmic Web Graph Entropy]{The cosmic web through the lens of graph entropy}

\author[Garc\'ia-Alvarado et al.]{
M. V. Garc\'ia-Alvarado,$^{1}$\thanks{E-mail: mv.garcia@uniandes.edu.co}
X.-D. Li$^{2}$
and J. E. Forero-Romero$^{1}$\thanks{E-mail: je.forero@uniandes.edu.co}
\\
% List of institutions
$^{1}$Departamento de F\'isica, Universidad de los Andes, Cra. 1 No. 18A-10 Edificio Ip, CP 111711, Bogot\'a, Colombia\\
$^{2}$School of Physics and Astronomy, Sun Yat-Sen University, Guangzhou 510297, P.R.China\\
}

% These dates will be filled out by the publisher
\date{Accepted XXX. Received YYY; in original form ZZZ}

% Enter the current year, for the copyright statements etc.
\pubyear{2020}

% Don't change these lines
\begin{document}
\label{firstpage}
\pagerange{\pageref{firstpage}--\pageref{lastpage}}
\maketitle

% Abstract of the paper
\begin{abstract}
  We explore the information theory entropy of a graph as a scalar to
  quantify the cosmic web. 
  We use simulations to gauge the influence of survey geometry, cosmic variance, 
  redshift space distortions, redshift evolution, cosmological parameters and 
  spatial number density.  
  Cosmic variance shows the least important influence while
  changes from the survey geometry, redshift space distortions, cosmological 
  parameters and redshift evolution produce larger changes on the
  order of $10^{-2}$ bits.
  The largest influence on the graph entropy comes from changes in the 
  number density of clustered points, with effects on the order of $0.2$ bits.
\textbf{The graph entropy can be
  used a discrete analogue of scalars that quantify the connectivity in
  continous density fields.}
  It is simple to compute and can be applied both to simulations and observational data 
  from large galaxy redshift surveys.
  This definition can be naturally applied to any graph built on other cosmological 
  probes such as cosmic microwave background maps, weak lensing maps or reconstructed
  density and peculiar velocity fields.
\end{abstract}
\begin{keywords}
cosmology: large-scale structure of Universe -- methods: data analysis.
\end{keywords}

%%%%%%%%%%%%%%%%%%%%%%%%%%%%%%%%%%%%%%%%%%%%%%%%%%

%%%%%%%%%%%%%%%%% BODY OF PAPER %%%%%%%%%%%%%%%%%%
\section{Introduction}

The cosmic web is one of the most salient features of the matter distribution
on cosmological scales. 
There is a great variety of methods striving to locally describe this complex 
structure by classifying a region as belonging either
to a void, filament, wall or cluster \citep{2018MNRAS.473.1195L}. 

Global descriptors are also useful to measure the topological structure of this web.
Quantities such as the topological invariants described by  the genus
\citep{1986ApJ...309....1H, 1986ApJ...306..341G} the Minkowski functionals 
\citep{1997ApJ...482L...1S}, the Betti numbers
\citep{2013JKAS...46..125P,2017MNRAS.465.4281P}, \textbf{or the average
number of filaments connected to a cluster \citep{2018MNRAS.479..973C}
} have been widely applied to describe the cosmic web. 

Another scalar, the information theory entropy based on the inhomogeneous 
spatial matter density in the cosmic web,
$S=-\int\rho(\bf{r})\log\rho(\bf{r}) d\bf{r}$, has
been used to derive analytical expressions for its cosmological evolution
\citep{2004PhRvL..92n1302H}, measurements from surveys
\citep{2015MNRAS.454.2647P}
and estimates from simulations \citep{2020MNRAS.491.5447V}.

The same information theory principles can be extended to measure the entropy built 
on graphs  used to describe the cosmic web. 
This entropy can be computed from the number of connections for each
point in the graph,  which is a property that  can be computed for commonly used graphs in
this context, such as  minimal spanning trees
\citep{1985MNRAS.216...17B}, Delaunay tessellations 
\citep{2007MNRAS.382....2R}, neighbor networks within a fixed linking
length \citep{2016MNRAS.459.2690H} or the $\beta$-skeleton
\citep{2019MNRAS.485.5276F}.   
For any of those graphs one can estimate the probability of a point
having $n$ connections, $P_n$, and from these values a global entropy
can be defined as $S = \sum_{P_n>0}-P_n\log_2{P_n}$. 

The purpose of this to letter is to advocate the information theory entropy on a graph
as a global scalar to describe the cosmic web.
We use the $\beta$-skeleton graph to explore a whole graph family 
by varying the real valued parameter $\beta$.
As cosmic web tracers we use dark matter halos from cosmological N-body simulations. 
This letter is structured as follows. 
In Section 2 we present the $\beta$-skeleton and the graph entropy definition.
In Section 3 we summarize the main features of the cosmological simulations we use and 
the basic setup of the numerical experiments we want to run.
The results are presented in Section 4.
We discuss the interpretation of our findings in Section
5 to finally summarize our Conclusions in Section 6.  

\section{Graphs and Entropy}

\subsection{The $\beta$-Skeleton Graph}

The $\beta$-Skeleton is a non-directed graph that determines the
connectivity between pairs of points as a function of the real
parameter $\beta$ \citep{1985Kirkpatrick}. 

To define whether a pair of points is connected by an edge, the
algorithm defines an exclusion region that changes with $\beta$.  
If the region is empty, then the two points are joined.
\textbf{We use the so-called \emph{lune} definition for this exclusion region.
The lune is defined as the intersection of two congruent spheres that
pass through the two points under consideration.}

\textbf{Figure \ref{fig:example} illustrates how the exclusion region changes
with $\beta$.
The diameter, $D$, of the spheres (circles in 2D) depends on the
distance between the two points, $d$, and the value of $\beta$; for
$\beta \leq 1$, $D=d/\beta$ and $D=d\beta$, otherwise.
The 1-Skeleton is also known as the Gabriel Graph, while the
$2$-skeleton corresponds to the Relative Neighbor Graph (RNG). }

\textbf{Under this definition the volume of the exclusion region is a
monotonic function of $\beta$.}
As the exclusion regions grows it is less likely to find pairs that are connected.
%Figure \ref{fig:connectionsBeta} shows the $\beta$-skeleton for a set of dark matter halos extracted from a cosmological simulation that are distributed inside an spherical shell.
The $\beta$-skeleton is a connected graph for $1\leq \beta\leq 2$
(i.e. all nodes have at least
one connection) and could be a disconnected graph for $\beta>2$ \citep{bose2002spanning}.
More details on the applications of this graph on large scale structure data
can be found in \citep{2019MNRAS.485.5276F}.

\subsection{Graph Entropy Definition}

After the graph has been constructed it is possible to estimate $P_n$, the probability of 
a node to have $n$ connections.
From these values we define the graph entropy as

\begin{equation}
\centering
    S = \displaystyle\sum_{P_n>0}-P_n\log_2{P_n}.
    \label{ecuacionEntropia}
\end{equation}
%
This graph entropy definition is the simplest possible as it is global and only 
takes into account the degree of connectivity \citep{2012Entrp..14..559M}. 
Figure \ref{fig:probabilities} shows some $P_n$ distributions for the 
$\beta$-skeleton graphs built on data from an N-body simulation.




\begin{figure}
    \centering
    \includegraphics[width=0.45\textwidth]{betas.pdf}
   \caption{To build the $\beta$-skeleton one has to define 
     the exclusion region between two points.
     This region is defined as the
     intersection between two congruent spheres that change with
     $\beta$ as shown in the Figure.
     The diameter of the spheres also changes with $\beta$ as
     described in the text.
 \label{fig:example}}  
\end{figure} 

%\begin{figure*}
%    \centering
%    \includegraphics[width=1.0\textwidth]{beta123_2.pdf}
%    \caption{$\beta$-Skeleton graph for a set of dark matter halos arranged over a shell
%    with inner radius $250$ \Mpch and outer radius $300$ \Mpch.
%    Each panel correspond to different values values of $\beta$ ($1$, $2$ and $\beta=3$).The case of $\beta=2$ corresponds to the Relative Neighbor Graph.
%    As $\beta$ increases links between nodes are lost. The probability of being a node with $n$ connections changes with $\beta$.}
%\label{fig:connectionsBeta}
%\end{figure*}

\begin{figure}
    \includegraphics[width=0.45\textwidth]{probabilities.pdf}
    \caption{Probabilities of having $n$ connections, $P_n$, for three different values of
    $\beta$. 
    The graph entropy summarizes the changes in the $P_n$ distribution as a function of $\beta$.}
    \label{fig:probabilities}
\end{figure}


\section{Methods}


\subsection{Simulations and Mock Catalogs}

We use data products from the Abacus project \citep{abacus}.
That project is a set of dark matter only N-body simulations that follows
the growth of structure in an explicit cosmological setup.
It includes simulations with different box sizes and cosmological parameters.

Here we focus on the simulations done a cubic box of $720$\Mpch on a side
with $1440^3$ particles. 
This resolution corresponds to a particle mass of $\sim1\times10^{10}$ \Msunh.
The same box was run with $20$ different initial conditions at a fixed value for
the cosmological parameters and also with fixed initial conditions and $40$ different
sets of cosmological parameters.

We use the Friend-of-Friends catalogs to build all the datasets to be used
as an input to the $\beta$-skeleton. In all cases we take only into account halos
with a maximum circular velocity larger than $300$\kms. 
We then apply different geometrical cuts and changes to their positions to build 
our final mock catalogs.

We build mock catalogs with two different geometries: spheres and spherical shells.
Both geometries have a maximum radius of $300$ \Mpch.
The shells have an inner radius of $250$\Mpch.
For each one of these geometries we build the corresponding random catalogs
by fixing their radial coordinate with respect to the geometrical center and 
randomize its angular coordinates.
We also produced catalogs with and without Redshift Space Distortion (RSD) effects. 
To study the effect of redshift evolution we use spheres extracted from
the snapshots at redshifts of $z=0.1$, $0.3$, $0.5$, $0.7$, $1$ and $1.5$.



\subsection{Numerical Experiments}

For each mock catalog we build the $\beta$-Skeleton and estimate each $P_n$ as  $\frac{N_e}{T_n}$, where $N_e$ is the number of nodes with $n$ connections and $N$ is the
total number of nodes in the data set.
From this set of $P_n$ values we measure the graph entropy using Equation \ref{ecuacionEntropia}.

We only use the values $\beta=1.0$, $1.5$, $2.0$, $2.5$, $3.0$, $3.5$, $4.0$, $4.5$ and $5$.
In what follows we preset results as a function of $\beta$ by continous lines.
We have checked that including more $\beta$ values does not significantly change
our results. 
In other words the graph entropy turns out to be a slowly varying function of $\beta$.


Our experiments focus on quantifying the effects on the entropy from:
cosmic variance, mock geometry, RSD, redshift evolution, cosmological
parameters and spatial number density. 


\begin{figure}
    \includegraphics[width=0.45\textwidth]{entropy.pdf}
    \caption{Graph entropy as a function of the parameter $\beta$ 
    for clustered and random points distributed inside a sphere. \label{fig:entropy}}
\end{figure}




\section{Results}
\label{sec:results}

\begin{figure*}
    \centering
    \includegraphics[width=0.41\textwidth]{cosmic_variance.pdf} 
    \includegraphics[width=0.41\textwidth]{geometry.pdf} 
    \includegraphics[width=0.41\textwidth]{rsd.pdf} 
    \includegraphics[width=0.41\textwidth]{redshift.pdf} 
    \includegraphics[width=0.41\textwidth]{param.pdf}
    \includegraphics[width=0.41\textwidth]{random_clustered_n1.pdf}
    \caption{Influence of different effects on the graph entropy. In all cases we 
    report the changes in entropy from a reference value. 
    Details are explained in the Results section. $a)$ Cosmic variance. 
    $b)$ Survey geometry. 
    $c)$ Redshift Space Distortions (RSD).
    $d)$ Redshift evolution.
    $e)$ Cosmological parameters. We only show the results for $\sigma_8$, 
    the parameter that shows the strongest correlation with the graph entropy.
    The plot corresponds to the entropy computed for the $1$-skeleton at $z=0.1$. 
    $f)$ Number density. 
    This calculation is performed on $20$ different spheres at $z=0.1$.
    A different percentage (shown in the caption) of these points are sampled. 
    We show the difference with respect to the entropy measured on the spheres 
    of sampled random points.\label{fig:diferencias}}
\end{figure*}    


Figure \ref{fig:entropy} shows the graph entropy as a function of $\beta$ for
clustered and random points distributed over a sphere.
The most important message from this plot is that the entropy can distinguish between
clustered and random points.
We see that the entropy follows a decreasing trend with increasing
$\beta$ that seems to be interrupted when $\beta=2$ (the RNG), which
represents a local minimum of the curve.  

This differentiates the point at which the constructed graph goes from
being fully connected, to have some isolated nodes.
\textbf{This change in the number of connections can be seen
  in Figure \ref{fig:probabilities}.
For $\beta\leq 2$ the probability of having $0$ connections is zero
(i.e. the graph is fully connected), while for $\beta>2$ that
probability is different from zero.}

For $\beta>2$ the entropy for the clustered points starts to
increase again, at a slower pace than it does for $\beta\leq 2$. 
Another interesting point is that around $\beta=5$ the two entropies
are equal.  
This point corresponds to the extreme of a very sparse graph where the
two different point distributions become indistinguishable from the
point of view of the entropy. 


The general shape shown in Figure \ref{fig:entropy} is conserved in
all cases.  
In what follows we quantify the entropy changes under different
conditions.

\subsection{Cosmic Variance}

We measure the effects of cosmic variance by comparing the entropy of the spherical 
mocks built from the $20$ different realizations of the standard
cosmology at $z=0.1$ without considering the effects of RSD.  
Panel $a)$ in Figure \ref{fig:diferencias} shows the entropy
difference with respect to a reference mock. The variance of the
entropy differences is on the order of $10^{-3}$ bits, almost
independent of $\beta$. 
The same result holds for mocks with shell geometry.



\subsection{Geometry}

We evaluate geometry influence by comparing spheres and shells.
We compare each spherical mock with its corresponding shell.
This is done at $z=0.1$ without RSD effects over the $20$ different realizations
of the standard cosmology.
Panel $b)$ in Figure \ref{fig:diferencias} shows that the entropy
difference has a strong dependence with $\beta$.  
For $\beta=1$ points on the spherical distribution have $8\times
10^{-2}$ more bits of entropy than shells. 
For values around $\beta=2$ this difference drops to negative values
(shells have more entropy than spheres) but only on the order of
$1\times10^{-2}$ bits. 




\subsection{Redshift Space Distortions}

We measured the RSD influence on the spheres extracted from the $20$ different realizations of the standard cosmology at $z=0.1$.
In the mock building process the RSD effect is applied before making the geometrical cut.
Panel $c)$ in Figure \ref{fig:diferencias} shows a similar picture than the one found in the previous subsection.
Namely, a strong dependence for $\beta\leq 2$, with more entropy in the data without RSD and a flat response for $\beta\geq 2$.
In this case the differences continue to be on the order of $10^{-2}$ bits, with the
largest difference located at $\beta=1$.


\subsection{Redshift Evolution}
We use six spheres to measure the redshift evolution of the graph entropy. 
Each sphere comes from a simulation with the standard cosmology with redshifts of
$z=0.1$, $0.3$, $0.5$, $0.7$, $1.0$ and $1.5$. 
We used the spheres in comoving coordinates without taking into account any form of RSD.
Panel $d)$ in Figure \ref{fig:diferencias} shows the entropy evolution computed for values
of $\beta=1$, $2$ and $3$.
For each $\beta$ we show the entropy differences with respect to the entropy at $z=0.1$.
Each $\beta$ value produces a different entropy redshift evolution.
The largest variation is $3\times 10^{-2}$ bits for the $1$-skeleton between the redshift of
$z=1.5$ and $z=0.1$, where the entropy decreases monotonously with decreasing redshift.
This redshift evolution is different for the $2$-skeleton and the $3$-skeleton; the changes 
are not monotonous and less pronounced.

\subsection{Cosmological Parameters}
The Abacus simulations have $40$ different realizations with different values for the
cosmological  parameters $H_0$, $\Omega_{DE}$, $\Omega_{M}$, $n_s$, $\sigma_8$ and $w_0$.
We measure the entropy differences with respect to the entropy for the cosmology of 
reference in the $40$ different spheres at $z=0.1$, without RSD effects, 
for the skeletons with $\beta=1$, $2$ and $3$.
Panel $e)$ in Figure \ref{fig:diferencias} 
shows the entropy differences as a function of $\sigma_8$ for $\beta=1$.
Increasing values of $\sigma_8$ correspond to lower entropy values. 
This is the strongest correlation we find between entropy and a cosmological parameter.
For $\beta=2$ the correlation is weaker and for $\beta=3$ the correlation inverses its
trend.
The different cosmological parameters induce changes on the entropy of magnitude $3\times10^{-2}$ bits at most.

\subsection{Number densities}

We prepare and additional dataset where we randomly 
sample a percentage (between $10\%$ and $100\%$ in jumps of $10\%$) 
of the points from the original random and clustered spheres.
Although the number density changes, the different points have the 
same correlation function as they are dominated by dark matter halos with circular velocities $\approx 300$ \kms.

The entropy changes in this case are the largest among our series of experiments.
Panel $e)$ in Figure \ref{fig:diferencias} shows the entropy differences 
with respect to the corresponding random sphere as a function
of $\beta$.
The entropy differences decrease with the number density. 
The differences range between $0.05$ to $0.2$ bits.

A central result is that the entropy only changes significantly in the
clustered points. 
If we compute the entropy on the spheres with random distribution of
points  with varying number densities, then the entropy changes among
those spheres  is only on the order of $10^{-2}$ bits. 

\section{Discussion}

\textbf{From all these findings, how can we then interpret the entropy measured on the
  $\beta$-skeleton graph built on LSS data?\\
\indent
Let us recall that the $\beta$-skeleton graph encodes topological
features from the data.
  Its entropy focuses on a single aspect: the connectivity distribution.
  In the $\beta$-skeleton the number of connection correlates with the
  point density.
  At fixed $\beta$, regions with larger number of connections are in 
  denser regions.
  This means that the connectivity distribution traces to some extent
  the point density distribution.
  The graph entropy at fixed $\beta$ can thus be thought as a summary
  statistic of the point density distribution.\\
\indent
  As $\beta$ increases, only the points in the densest regions remain
  connected \citep{2019MNRAS.485.5276F}.
  This means that the entropy as a function of $\beta$ describes the
  balance between the number of 
  overdense regions in the point distribution (which are still
  connected) and the low density regions (that will be disconnected
  for large values of $\beta$.)\\
\indent
As a complementary test of this reasoning we computed the Voronoi
tesellation over the datasets used in this Letter. 
We find that the distribution of the Voronoi cell volumes (a measure
of point density; large volumes can be interpreted as low density regions) 
has similar properties as the graph entropy; it is invariant for random points,
weakly dependent on cosmic variance, geometry, RSD, redshift and
cosmological parameters and strongly dependent on the number density
of clustered points.
Furthermore, as $\beta$ increases, the cell volume distributions for the points included in
the graph selects mostly regions of large density (small volumes.)\\
\indent
A complementary point of view is to consider that the local density
correlates with cosmic web features,
i.e. cluster-like regions are overdense, voids are underdense, while filaments and sheets have
 intermediate densities \citep{2018MNRAS.473.1195L}. 
 Changes in the number density clearly impact the cosmic web defined by the tracers.
  As the number density decreases, we observe how the voids become
  larger and the filamentary patterns are less pronounced.  
  To quantify this visual impression we estimate the typical void size
  (in units of mean interparticle distance)
  in the datasets with changing number density.  
  We find that for clustered points this quantity increases with
  decreasing number density and stays constant for random points,
  as expected.
  In clustered data, larger entropy values correspond to smaller voids.\\
\indent
From these lines of reasoning we argue that the graph entropy as a
  function of $\beta$ can be seen as a the discrete analgoue of
  the measuring scalar topological invariants 
  over a continous density field with varying density threholds \citep{2013JKAS...46..125P}.}



\section{Conclusions}

In this letter we presented the graph entropy as a new scalar to 
quantify the large scale structure of the Universe.
The entropy definitions we use is a global quantity based on 
the definition commonly used in information theory.
It uses the probability of having $n$ connections, $P_n$, to build the 
scalar $S=\sum_{P_n>0} -P_n  \log_2 P_n$. 
We based our quantitative analysis on the $\beta$-skeleton graph 
built on mock catalogs constructed from cosmological N-body simulations.

We measured that the graph entropy is of order unity and ranges 
between $1.6$ and $3.2$ bits.
Then we tested the influence on the graph entropy of six major factors: 
cosmic variance, survey geometry, redshift space distortions, redshift evolution, cosmological parameters and number densities.
We found that the strongest influence on the entropy appears in
clustered data at different number  densities.

Qualitatively speaking \textbf{the graph entropy as a
  function of $\beta$ can be thought as the discrete analogue
  of measuring the connectivity of a continous density field with
  varying density threholds}. 

Using the graph entropy to disentangle different cosmological effects
seems to be a challenging task. 
The influence of survey geometry, redshift evolution and cosmological parameters 
all seem to be on the same order of magnitude when measured in the entropy.
One way to make stand out an effect over the others (for instance, the influence of
$\sigma_8$ over RSD) might be comparing the results of clustered measurements against
a random counterpart with the same geometry and number density.

Further tests of the applicability to observational data and other graphs 
are necessary beyond the  promising results presented here.
An incomplete list of experiments that might complement its usage in the context of
cosmology are: using this entropy to check for isotropy (e.g. measuring the entropy from
different parts of the sky); measuring the entropy in different modified gravity theories;
building graphs and measuring the entropy from 
other quantities, i.e. features in the Cosmic Microwave Background, weak lensing peaks,
basin points from reconstructed peculiar velocity fields.

\section*{Acknowledgements}
XDL acknowledges the support from NSFC grant (No. 11803094).

\bibliographystyle{mnras}
\bibliography{references}

%\bsp	% typesetting comment
%\label{lastpage}
\end{document}
